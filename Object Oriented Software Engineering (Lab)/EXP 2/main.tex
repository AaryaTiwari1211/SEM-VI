\documentclass{article}

\usepackage[utf8]{inputenc}
\usepackage{geometry}
\usepackage{graphicx}
\usepackage{float}
\usepackage{hyperref}

\geometry{
    a4paper,
    total={170mm,257mm},
    left=20mm,
    top=20mm,
}

\title{Software Project Management Plan (SPMP)\\Remote Wildlife Monitoring and Conservation Application}
\author{Aarya Tiwari and Dhairya Satra}
\date{\today}

\begin{document}

\maketitle

\tableofcontents
\newpage

\section{Introduction}

\subsection{Project Overview}

The Remote Wildlife Monitoring and Conservation Application aims to address the increasing threats to wildlife habitats by leveraging technology for habitat monitoring, wildlife tracking, and environmental assessment. This SPMP outlines the project management approach for its development.

\subsection{Objectives}

\begin{itemize}
    \item Develop a comprehensive Remote Wildlife Monitoring and Conservation Application.
    \item Implement Aerial Surveillance and Habitat Monitoring module.
    \item Enhance wildlife monitoring, conservation management, research, education, and stakeholder engagement.
\end{itemize}

\section{Project Scope}

The project includes:

\begin{itemize}
    \item Aerial surveillance using drones.
    \item Installation of camera traps and sensor networks.
    \item Data collection, analysis, visualization, and reporting functionalities.
    \item Collaboration and stakeholder engagement initiatives.
    \item Education and outreach programs.
\end{itemize}

\section{Choice of Process Model}

For the development, the Iterative and Incremental Development Model will be employed, allowing continuous improvement, flexibility, and adaptability.

\section{Roles and Responsibilities}

\begin{itemize}
    \item \textbf{Project Manager:} Overall coordination, planning, execution, monitoring, and control.
    \item \textbf{Technical Lead:} Architectural design, technology selection, system integration, and quality assurance.
    \item \textbf{Data Scientist:} Data collection, preprocessing, analysis, modeling, visualization, and decision-making.
    \item \textbf{GIS Specialist:} Spatial data management, mapping, and environmental planning.
    \item \textbf{Drone Operators:} Aerial surveillance, data capture, and equipment maintenance.
    \item \textbf{Field Researchers:} Camera trap deployment, wildlife monitoring, and data collection.
    \item \textbf{Community Engagement Officer:} Stakeholder collaboration, capacity building, and communication.
    \item \textbf{UI/UX Designer:} User interface design, usability testing, and visual aesthetics.
    \item \textbf{Quality Assurance Team:} Testing, validation, bug tracking, and quality control.
\end{itemize}

\section{Module: Aerial Surveillance and Habitat Monitoring}

\subsection{Functionality}

\begin{itemize}
    \item Display live drone footage, images, and sensor readings.
    \item Enable users to navigate, zoom, pan, and interact with aerial maps.
    \item Provide access to historical data, trends, patterns, alerts, and notifications.
    \item Allow users to configure settings, filters, parameters, and preferences.
\end{itemize}

\subsection{Objectives}

\begin{itemize}
    \item Monitor wildlife habitats.
    \item Assess environmental conditions.
    \item Track endangered species.
    \item Identify conservation priorities.
\end{itemize}

\subsection{Benefits}

Enhanced user engagement, data accessibility, visualization capabilities, and decision-making support for conservation initiatives.

\section{Conclusion}

This SPMP provides a framework for the successful development of the Remote Wildlife Monitoring and Conservation Application. It outlines the project's objectives, scope, choice of process model, roles and responsibilities, and specific details about the Aerial Surveillance and Habitat Monitoring module.

\end{document}
