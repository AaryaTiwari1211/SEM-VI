\documentclass{article}
\usepackage{graphicx}

\title{Software Test Document (STD)}
\author{}
\date{}

\begin{document}
\maketitle

\section{Introduction}

\subsection{System Overview}
Briefly detail the software system and items to be tested. Identify the version(s) of the software to be tested.

\subsection{Test Approach}
Describe the overall approach to testing. Specify major activities, techniques, and tools used. Identify significant constraints on testing.

\section{Test Plan}

Describe the scope, approach, resources, and schedule of testing activities. Identify items being tested, features to be tested, testing tasks, and responsible personnel.

\subsection{Features to be Tested}

Identify all software features and combinations of features to be tested. Identify associated test cases and software versions.

\subsection{Features not to be Tested}

Identify features and combinations not to be tested and reasons for exclusion.

\subsection{Testing Tools and Environment}

Specify test staffing needs and requirements of the test environment.

\section{Test Cases}

\subsection{Case-n (use a unique ID of the form TC-nnnn for this heading)}

\subsubsection{Purpose}

Identify the version of the software and test items. Describe features and reference associated requirements.

\subsubsection{Inputs}

Specify inputs required for test case execution.

\subsubsection{Expected Outputs \& Pass/Fail Criteria}

Specify expected outputs and pass/fail criteria for test items.

\subsubsection{Test Procedure}

Detail procedures needed to execute the test case.

\section{Test Logs}

\subsection{Log for test-n (use a unique ID of the form TL-nnnn for this heading)}

\section{Test Results}

Record date/time and observed results for each execution. Indicate successful or unsuccessful execution.

\section{Incident Report (add a unique ID of the form TIR-nnnn to this heading)}

Summarize incidents, identifying test items involved and anomalies in results.

\end{document}
